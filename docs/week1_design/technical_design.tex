\documentclass[11pt]{article}
\usepackage[margin=1in]{geometry}
\usepackage{enumitem}
\usepackage{hyperref}
\title{Technical Design Document\\Robotics Path Finder}
\author{Explorer Track -- Coding Caf\'e}
\date{November 2025}
\begin{document}
\maketitle

\section{Project Overview}
The goal is to prototype a path-finding tool that consumes publicly available LiDAR datasets, extracts traversable terrain, and plans feasible routes for a ground robot. The near-term prototype emphasises data handling, environment modelling, and search-based planning while keeping the user-facing interface minimal.

\section{System Architecture}
The planned solution follows a modular pipeline so that each stage can evolve independently as the project matures.
\begin{enumerate}[label=\textbf{\arabic*.}]
    \item \textbf{Data Acquisition Layer}: Handles discovery and download of candidate LiDAR tiles from the USGS data portal. Interfaces with filesystem/object storage for caching.
    \item \textbf{Pre-processing Layer}: Normalises raw point clouds, clips regions of interest, and generates raster products (elevation grids, slope maps) suitable for planning.
    \item \textbf{Environment Representation}: Converts processed rasters into a 2D cost map that encodes traversability constraints, robot footprint limits, and waypoint metadata.
    \item \textbf{Path Planning Core}: Implements grid-based search (baseline: A*) with hooks for heuristics, tie-breaking, and constraint tuning.
    \item \textbf{Visualization \\ Reporting}: Provides quick-look plots/notebooks to interpret candidate paths and validate assumptions before committing to field tests.
\end{enumerate}

\section{Data Flow}
\begin{enumerate}[label=\textbf{Step \arabic*.}]
    \item User selects an area of interest; the acquisition layer fetches the corresponding LAS/LAZ tiles and persists them locally.
    \item The preprocessing layer batches tiles, filters outliers, normalises elevation, and generates raster grids.
    \item The environment representation module fuses the grids, applies masking rules (water, steep gradients, restricted zones), and emits a planning-ready cost map.
    \item The planner consumes the cost map, applies robot-specific constraints, and produces a ranked list of candidate paths with metadata (path length, elevation gain).
    \item Visualization notebooks or lightweight scripts summarise outputs for iteration and validation with mentors.
\end{enumerate}

\section{Core Algorithms}
\subsection{Search Strategy}
A* search on a regular grid is the baseline to guarantee deterministic behaviour and provide a reference point for later algorithmic improvements (e.g., weighted A*, D* Lite, or sampling-based planners). Heuristic design will balance computational cost with terrain fidelity so early tests remain responsive on commodity hardware.

\subsection{Cost Function (Reserved)}
\textit{Reserved space: detailed derivation and tuning strategy for the cost function will be added after internal review.}

\section{Interfaces \\ Extensibility}
\begin{itemize}[leftmargin=*]
    \item \textbf{Module Contracts}: Each layer exchanges data through structured files (GeoTIFF rasters, JSON metadata packs) to keep boundaries explicit and testable.
    \item \textbf{Configuration}: Centralised YAML configuration will describe tile IDs, robot parameters, and planner toggles so experiments remain reproducible.
    \item \textbf{Future Hooks}: Slots are left for incremental upgrades such as GPU-accelerated raster processing or ROS integration without major refactors.
\end{itemize}

\section{Learning Objectives}
\begin{itemize}[leftmargin=*]
    \item \textbf{New Territory}: Raster-based terrain classification, scalable LiDAR preprocessing pipelines, and custom heuristic design for non-holonomic robots.
    \item \textbf{Reinforcement}: Applying A* variants, structuring data pipelines, and documenting experiments in a reproducible way.
    \item \textbf{Comfort Zone}: Python scripting, NumPy/Pandas workflows, and exploratory analysis inside Jupyter notebooks.
\end{itemize}

\section{Risks \& Mitigations}
\begin{itemize}[leftmargin=*]
    \item \textbf{Data Volume}: Large LiDAR tiles may exceed local storage. Mitigation: staged downloads, coarse subsampling for early tests.
    \item \textbf{Terrain Fidelity}: Raster simplifications could hide hazards. Mitigation: integrate gradient checks and manual validation plots.
    \item \textbf{Time Constraints}: Balancing preprocessing and planner development. Mitigation: prioritise baseline pipeline first, defer optimisations.
\end{itemize}

\end{document}
