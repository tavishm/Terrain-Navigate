\documentclass{article}
\usepackage[utf8]{inputenc}
\usepackage[T1]{fontenc}
\title{Technology Stack}
\author{Tavish Mankash}
\date{}
\begin{document}
\maketitle
\small
\section*{Core}
\begin{itemize}
  \item Python 3.11+ (speed, typing, ecosystem)
  \item NumPy (arrays, vectorized slope math)
  \item (Later) PDAL or \texttt{laspy} for LAZ ingest
  \item (Later) SciPy: KD-tree / interpolation utilities
\end{itemize}

\section*{Dev + Quality}
\begin{itemize}
  \item \texttt{pytest} minimal unit tests
  \item \texttt{ruff} / \texttt{black} (style; optional early) 
\end{itemize}

\section*{Visualization (optional early)}
\begin{itemize}
  \item Matplotlib: quick 2D plots of height + path
  \item (Possible) Plotly: interactive 3D surface when needed
\end{itemize}

\section*{Performance Path}
Start pure Python/NumPy. If hotspots appear:
\begin{itemize}
  \item Numba JIT for inner loops
  \item PyTorch or JAX only if GPU batch path planning emerges
\end{itemize}

\section*{Packaging}
Simple src layout; no heavy framework. One \texttt{__main__} entry for demos.

\section*{Data}
Local cache directory (e.g. \texttt{data/raw}, \texttt{data/processed}). Avoid bundling big files in git.

\section*{Why This Stack}
Fast iteration, minimal friction, easy to grow to heavier geo libs later.

\end{document}
